\sang{Der er et yndigt Land}
{Musik: Hans Ernst Krøyer}
{Tekst: Adam Oehlenschläger}
{
Der er et yndigt Land,
Det staaer med brede Bøge
Nær salten Østerstrand;
Det bugter sig i Bakke, Dal,
Det hedder gamle Danmark,
Og det er Freias Sal.

Der sad i fordums Tid
De harniskklædte Kæmper,
Udhvilede fra Strid;
Saa drog de frem til Fienders Meen,
Nu hvile deres Bene
Bag Høiens Bautasteen.

Det Land endnu er skiønt,
Thi blaa sig Søen belter,
Og Løvet staaer saa grønt;
Og ædle Qvinder, skiønne Møer,
Og Mænd og raske Svende
Beboe de Danske Øer.

Vort Sprog er stærkt og blødt,
Vor Tro er reen og luttret
Og Modet er ei dødt.
Og hver en Dansk er lige fri,
Hver lyder tro sin Konge,
Men Trældom er forbi.

Et venligt Syd i Nord
Er, grønne Danarige,
Din axbeklædte Jord.
Og Snekken gaaer sin stolte Vei.
Hvor Ploug og Kiølen furer,
Der svigter Haabet ei.

Vort Dannebrog er smukt,
Det vifter hen ad Havet
Med Flagets røde Bugt.
Og stedse har sin Farve hvid
Dit hellige Kors i Blodet,
O Dannebrog, i Strid.

Karsk er den Danskes Aand,
Den hader Fordoms Lænker,
Og Sværmeriets Baand.
For Venskab aaben, kold for Spot,
Slaaer ærlig Jydes Hierte,
For Pige, Land og Drot.

Jeg bytter Danmark ei,
For Ruslands Vinterørkner,
For Sydens Blomstermai.
Ei Pest og Slanger kiende vi,
Ei Vesterlandets Tungsind,
Ei Østens Raseri.

Vor Tid ei staaer i Dunst,
Den hævet har sin Stemme
For Videnskab og Kunst.
Ei Bragis og ei Mimers Raab
Har vakt i lige Strækning
Et bedre Fremtids Haab.

Ei stor, vor Fødestavn,
\ \ Dog hæver sig blandt Stæder
\ \ Dit stolte Kiøbenhavn.
\ \ Til bedre By ei Havet kom,
\ \ Ja ingen Flod i Dalen,
\ \ Fra Trondhiem og til Rom.

Med hellig Varetægt
\ \ Bevare du, Alfader!
\ \ Vor gamle Kongeslægt.
\ \ Kong Fredrik ligner Fredegod;
\ \ Hvor er en bedre Fyrste,
\ \ Af bedre Helteblod?

Hil Drot og Fædreland!
\ \ Hil hver en Danneborger,
\ \ Som virker hvad han kan.
\ \ Vort gamle Danmark skal bestaae,
\ \ Saalænge Bøgen speiler
\ \ Sin Top i Bølgen blaa.
}