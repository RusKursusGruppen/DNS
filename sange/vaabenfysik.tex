\sang{Hey ho for våbenfysik}
{FysikRevy\tm 2003}
{Kim Larsen - Jutlandia}
{
Det var i 1945 og
nu ville man ha' fred,
men der var krig i Japan.
Der blev samlet uran nok,
og det blev kastet ned,
for der var krig i Japan.
De fik at se, hvad fysiken formår
og sjove børn de næste mange år.

\omkv{
Hey Ho for våbenfysik!
Vi blæser på alle trakt-a-ter.
Bomber her, bomber der, bomber for fred!
Hvad skal vi med diplomater?
}

Vi flyver gennem natten
og gir' din fjende smæk.
Han ser os ikke komme.
Vi stiller ingen spørgsmål,
og så snart vi har din check,
så er krigen omme.
Hvis du gi'r, fyrer vi den af.
Det' jo det, de unge vil ha'.

\omkv{
Hey Ho ...
}

Vores salgskontor har åbent hele døgnet,
bare ring. 
Du skal ikke tøve.
Vacuum, brint og EMP
og andre sjove ting,
dem vil vi gerne prøve.
Her er ugens tilbudskatalog:
Start tre krige og betal for to.

\omkv{
Hey Ho ...
}

\omkv{
Hey Ho for våbenfysik!
Vi blæser på alle trakt-a-ter.
Bomber her, bomber der, bomber for fred!
Vi nakker de slyngelstater!
}

\kom{Der findes svarvers til denne sang. Få din vejleder til at synge dem.}

% Svarvers, så det ikke går tabt
%
% Det var i Køge sidste sommer, eller cirka deromkring
% og vi sku' ud og lege.
% Vi var klædt i flyverdragt, med alle vores ting
% og vi var rigtig seje.
% Udstyret, med skovl og med spand
% nu skal vi lege i det våde sa-a-a-and

% Hey ho, for våbenfysik!
% Det leger vi på legepladsen
% Bomber her, bomber der, bomber af SAND!
% De hober sig op i sandkassen.

}